\documentclass[dvipdfmx,uplatex]{jsarticle}
\title{知識と情報科学第7,8,9,10回}
\author{長田悠生}
\date{2023/7/13}

\usepackage{url}
\usepackage[dvipdfmx]{graphicx}

\begin{document}
  \begin{titlepage}
    \maketitle
    \begin{center}
      \textmc{\HUGE \LaTeX}
    \end{center}
    \thispagestyle{empty}
  \end{titlepage}

  \textmc{\LARGE 第7回の要約\\}
  \textmc{
    この講義は、生物行動についてだった。生物行動を明らかにするために、情報科学は必要不可欠である。
    行動をデータとして定量化して数理モデリングすることで、生物行動を表現できる。
    この講義では、カエルの生物行動についての説明だった。
    まず、複数の同種のカエルが鳴く際の互いの鳴き声のずれについての説明だった。
    オスのカエルは、メスを呼ぶために鳴くことが知られている。
    同種のオスのカエルを複数集めたときに、それらのカエルは互いにどのような周期のずれで鳴くかを調査した。
    調査の結果、二匹のカエルの場合は、逆相同期現象が起こることがわかった。
    この現象について数理化を行っても、位相が$\pi$ずれていることが示せた。
    次に、三匹で実験を行った。この場合は、2匹が同期してなくバターンと3匹が三相同期してなくパターンが観測された。
    この現象について数理化した場合も、同様にこれらの現象を示すことができた。
    これを十から二十匹の規模に拡大しても、同様の結果を得ることができた。
    現象をシンプルなモデルにモデル化して考えることで、現象について定式的に説明できるのだ。
    次の実験は、カエルの鳴き声に寄ってくる寄生者に関する実験だった。
    この実験によって得られた結果についても、数理化することでどのようなオスがより寄生されやすいかを示せた。
    生物行動の研究は、データ計測$\to$数理モデルの構築$\to$信号処理における実測値の解析のサイクルを回すことで行われている。
  \\}
  \textmc{\large \\}
  \textmc{\LARGE 第8回の要約\\}
  \textmc{
  \\}
  \textmc{\large \\}
  \textmc{\LARGE 第9回の要約\\}
  \textmc{
  \\}
  \textmc{\large \\}
  \textmc{\LARGE 第10回の要約\\}
  \textmc{
  \\}
  \textmc{\large \\}
  \textmc{\LARGE 第7回の講義についてのキーワード: \\}
  \textmc{
  \\}
  \textmc{\large \\}
  \textmc{\LARGE 第8回の講義についてのキーワード: \\}
  \textmc{
  \\}
  \textmc{\large \\}
  \textmc{\LARGE 第9回の講義についてのキーワード: \\}
  \textmc{
  \\}
  \textmc{\large \\}
  \textmc{\LARGE 第10回の講義についてのキーワード: \\}
  \textmc{
  \\}

\end{document}