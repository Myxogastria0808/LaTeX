\documentclass[dvipdfmx,uplatex]{jsarticle}
\title{確率論第4回課題}
\author{長田悠生}
\date{2023/5/21}

\usepackage{amsmath,amssymb}
\usepackage{signchart}

\begin{document}
  \begin{titlepage}
    \maketitle
    \begin{center}
      \textmc{\HUGE \LaTeX}
    \end{center}
    \thispagestyle{empty}
  \end{titlepage}

    \centerline{\huge 演習課題4A}
    \vspace{10mm}
    \textmc{${p}_{X,Y}(x,y)$について表にまとめる。\\}
    \begin{table}[htb]
    \centering
        \caption{${p}_{X,Y}(x,y)$}
        \begin{tabular}{|c||c|c|c|c|c|} \hline
            $4$ & $\frac{4!}{4!0!0!} (\frac{1}{6})^4 (\frac{1}{6})^0 (\frac{4}{6})^0$ & $0$ & $0$ & $0$ & $0$ \\ \hline
            $3$ & $\frac{4!}{3!0!1!} (\frac{1}{6})^3 (\frac{1}{6})^0 (\frac{4}{6})^1$ & $\frac{4!}{3!1!0!} (\frac{1}{6})^3 (\frac{1}{6})^1 (\frac{4}{6})^0$ & $0$ & $0$ & $0$ \\ \hline
            $2$ & $\frac{4!}{2!0!2!} (\frac{1}{6})^2 (\frac{1}{6})^0 (\frac{4}{6})^2$ & $\frac{4!}{2!1!1!} (\frac{1}{6})^2 (\frac{1}{6})^1 (\frac{4}{6})^1$ & $\frac{4!}{2!2!0!} (\frac{1}{6})^2 (\frac{1}{6})^2 (\frac{4}{6})^0$ & $0$ & $0$ \\ \hline
            $1$ & $\frac{4!}{1!0!3!} (\frac{1}{6})^1 (\frac{1}{6})^0 (\frac{4}{6})^3$ & $\frac{4!}{1!1!2!} (\frac{1}{6})^1 (\frac{1}{6})^1 (\frac{4}{6})^2$ & $\frac{4!}{2!1!1!} (\frac{1}{6})^2 (\frac{1}{6})^1 (\frac{4}{6})^1$ & $\frac{4!}{3!1!0!} (\frac{1}{6})^3 (\frac{1}{6})^1 (\frac{4}{6})^0$ & $0$ \\ \hline
            $0$ & $\frac{4!}{0!0!4!} (\frac{1}{6})^0 (\frac{1}{6})^0 (\frac{4}{6})^4$ & $\frac{4!}{1!0!3!} (\frac{1}{6})^1 (\frac{1}{6})^0 (\frac{4}{6})^3$ & $\frac{4!}{2!0!2!} (\frac{1}{6})^2 (\frac{1}{6})^0 (\frac{4}{6})^2$ & $\frac{4!}{3!0!1!} (\frac{1}{6})^3 (\frac{1}{6})^0 (\frac{4}{6})^1$ & $\frac{4!}{4!0!0!} (\frac{1}{6})^4 (\frac{1}{6})^0 (\frac{4}{6})^0$ \\ \hline \hline
            $Y/X$ & $0$ & $1$ & $2$ & $3$ & $4$ \\ \hline
        \end{tabular}
    \end{table} \\
    \textmc{表より、同時分布${p}_{X,Y}(x,y)$は、\\}

    \begin{equation}
        \therefore {p}_{X,Y}(x,y) = 
            \left \{
                \begin{aligned}
                    &\hspace{5mm} \frac{1}{1296} &if(x,y)=(4,0),(0,4) \\
                    &\hspace{5mm} \frac{1}{81} &if(x,y)=(3,0),(0,3) \\
                    &\hspace{5mm} \frac{2}{27} &if(x,y)=(2,0),(0,2) \\
                    &\hspace{5mm} \frac{16}{81} &if(x,y)=(1,0),(0,1)(0,0) \\
                    &\hspace{5mm} \frac{1}{324} &if(x,y)=(3,1),(1,3) \\
                    &\hspace{5mm} \frac{1}{27} &if(x,y)=(2,1),(1,2) \\
                    &\hspace{5mm} \frac{4}{27} &if(x,y)=(1,1) \\
                    &\hspace{5mm} \frac{1}{216} &if(x,y)=(2,2) \\
                \end{aligned}
            \right. \nonumber
    \end{equation}

    \newpage
    \centerline{\huge 演習課題4B}
    \vspace{10mm}
    \textmc{まず、Zの値がkのときの様子を以下に示す。 \\}
    \textmc{以下の数直線は、Zのポアソン分布を表している数直線 \\}
    \signchart[width=10,height=1,arrows=|->]{0(最小値),1,2,…,k,k+1,…}{,,,,,,,動物の数の総和}
    \vspace{7mm}
    \textmc{さらに、kについて1日をk等分した数直線上にウサギを1羽ずつ確率pで存在させる様子を数直線で表すと、以下のようになる。\\}
    \signchart[width=15,height=1,arrows=|<->|]{,,…,no rabbit,…,a rabbit,a rabbit,…,}{,,,,,,,,,数直線全体が1日を表す}
    \textmc{
        1日を$k$当分したとすると、区間の数は$k$、ウサギの頭数を$\alpha$として、1つの区間にウサギが1羽いるかいないかとなるようにウサギを区間に分配したとすると、
        各区間にウサギが分配されている確率は$p$なので、$k$個の区間の内、ウサギがいる区間の個数の確率は、
    \\}
    \begin{equation}
        {p}_{k \cap X}(\alpha) = \frac{\lambda^k e^{-\lambda}}{k!} \times {}_{k}\!\mathrm{C}_{\alpha} p^\alpha (1-p)^{k-\alpha}
    \end{equation}
    textmc{${p}_{X}(x)$がポアソン分布に従っていれば良いので、}
    \begin{equation}
        {p}_{X}(x) = e^{-\lambda p} \times \frac{(\lambda p)^x}{x!}
    \end{equation}
    \textmc{を示せばよい。 \\}
    \textmc{
        ${p}_{X}(x)$の値をある値$x$に固定した場合、${p}_{X}(x)$の値は$\sum_{k=x}^\infty {p}_{k \cap x}(\alpha)$である。
        また、$X=x$のときの$y$の値は0から$\infty$まで取り得るので、$X=x$のすべての場合の総和を$\sum_{k=x}^\infty$で表す。
        これは、$X=x$における$Z=x+y$のすべての場合の総和を表していることと同等である。 
    \\}
    \textmc{よって${p}_{X}(x)$は、以下の式で表せる。}
    \begin{equation}
        {p}_{X}(x) = \sum_{k=x}^\infty {p}_{X,Y,Z}(x,y,x+y)
    \end{equation}
    \textmc{上記の式を次ページのように変形していく。 \\}
    \begin{align}
        {p}_{X}(x) &= \sum_{k=x}^\infty \frac{\lambda^{x+y} e^{-\lambda}}{(x+y)!} \times {}_{x+y}\!\mathrm{C}_{x} p^x (1-p)^y \nonumber \\
        &= \sum_{k=x}^\infty \frac{\lambda^{x+y} e^{-\lambda}}{(x+y)!} \times \frac{(x+y)!}{{(x+y)-x}x!} p^x (1-p)y \nonumber \\
        &= \frac{p^x e^{-\lambda} \lambda^x}{x!} \times \sum_{k=x}^\infty \frac{\lambda^y \times (1-p)^y}{y!} \nonumber \\
        &= \frac{p^x e^{-\lambda} \lambda^x}{x!} \times \sum_{k=x}^\infty \frac{(\lambda - \lambda p)^y}{y!} \nonumber \\
        &= \frac{p^x e^{-\lambda} \lambda^x}{x!} \times e^{\lambda - \lambda p} \nonumber \\
        &= e^{-\lambda p} \times \frac{(\lambda p)^x}{x!} \nonumber
    \end{align}
    \vspace{7mm}
    \textmc{${p}_{Y}(y)$も同様に、${p}_{Y}(y) = e^{-\lambda(1-p)} \times \frac{ \left\{ \lambda(1-p) \right\}^y}{y!}$となる。\\}
    \textmc{$\therefore$ $X~Poisson(\lambda p)$、$Y~Poisson(\lambda(1-p))$は、成り立つ。}





\end{document}
