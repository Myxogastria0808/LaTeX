\documentclass[dvipdfmx,uplatex]{jsarticle}
\title{力学1演習課題【4】}
\author{
    名前: 長田悠生\\
    学籍番号: 202310330\\
}
\date{2023/6/18}

\usepackage{amsmath,amssymb}
\usepackage{bm}
\usepackage{mathtools}

\begin{document}
  \begin{titlepage}
    \maketitle
    \begin{center}
      \textmc{\HUGE \LaTeX}
    \end{center}
    \thispagestyle{empty}
  \end{titlepage}

  \centerline{\LARGE 4-2}
  \vspace{10mm}
  \textmc{\large (1)\\}
  \begin{equation}
    \begin{aligned}
        &\bm{F} = - m\omega^2x(t)より、\nonumber\\
        &\bm{F} = m\frac{d^2}{dt^2}x(t) = -m\omega^2x(t)\nonumber\\
        &\therefore m\frac{d^2}{dt^2}x(t) + m\omega^2x(t) = 0\nonumber\\
    \end{aligned}
  \end{equation}

  \textmc{\large (2-1)\\}
  \begin{equation}
    \begin{aligned}
        &x(t) = e^{\lambda t}\nonumber\\
        &\therefore \frac{d^2}{dt^2}e^{\lambda t} + \omega^2e^{\lambda t} = 0\nonumber\\
    \end{aligned}
  \end{equation}

  \textmc{\large (2-2)\\}
  \begin{equation}
    \begin{aligned}
        &x(t) = e^{\lambda t}\nonumber\\
        &\frac{d^2}{dt^2}e^{\lambda t} + \omega^2e^{\lambda t} = 0\nonumber\\
        &\lambda^2e^{\lambda t} + \omega^2e^{\lambda t} = 0\nonumber\\
        &e^{\lambda t}(\omega^2 + \lambda^2) = 0\nonumber\\
        &e^{\lambda t} > 0より、\nonumber\\
        &\lambda^2 = -\omega^2\nonumber\\
        &\lambda = \pm \omega i\nonumber\\
        &\therefore {x}_{1}(t) = e^{\omega t i}、{x}_{2}(t) = e^{-\omega t i}\nonumber\\
    \end{aligned}
  \end{equation}

  \textmc{\large (2-3)\\}
  \begin{equation}
    \begin{aligned}
        &e^{\omega t i} \times A = e^{-\omega t i} \hspace{3mm} (Aは定数関数)とおくと、\nonumber\\
        &A=e^{-2\omega t i}\nonumber\\
        &\therefore Aは定数関数ではないので、x_{1}(t)とx_{2}(t)は一次独立\nonumber\\
        &\therefore (一般解) x(t) = Ae^{\omega t i} + Be^{-\omega t i}\nonumber\\
    \end{aligned}
  \end{equation}

  \textmc{\large (3)\\}
  \begin{equation}
    \begin{aligned}
        &x(t) = Ae^{\omega t i} + Be^{-\omega t i}\nonumber\\
        &=A(\cos \omega t + i \sin \omega t) + B{\cos(-\omega t) + i \sin (-\omega t)}\nonumber\\
        &=A(\cos \omega t + i \sin \omega t) + B(\cos \omega t -i \sin \omega t)\nonumber\\
        &\therefore A(\cos \omega t + i \sin \omega t) + B(\cos \omega t -i \sin \omega t)\nonumber\\
    \end{aligned}
  \end{equation}

  \newpage
  \textmc{\large (4)\\}
  \begin{equation}
    \begin{aligned}
        &x(t) = A(\cos \omega t + i \sin \omega t) + B(\cos \omega t -i \sin \omega t)\nonumber\\
        &=(A+B)\cos \omega t + i(A-B)\sin \omega t\nonumber\\
        &t=0のとき、\nonumber\\
        &x(0)=(A+B)\cos0 + i(A-B)\sin0\nonumber\\
        &=A+B\nonumber\\
        &このとき、x(0)=Dより、A+B=Dである。\nonumber\\
        &v(t)=\frac{d}{dt}x(t) = -\omega(A+B)\sin \omega t + \omega(A-B)i\cos \omega t\nonumber\\
        &t=0のとき、\nonumber\\
        &v(0)=-\omega(A+B)\sin0 + \omega(A-B)i\cos0=0\nonumber\\
        &\omega(A-B)i=0\nonumber\\
        &\omega \neq 0より、\nonumber\\
        &A-B=0\nonumber\\
        &A=B=\frac{1}{2}D\nonumber\\
        &\therefore x(t) = D\cos\omega t\nonumber\\
        &おもりの運動は、振幅Dの \cos 波になる。\nonumber\\
    \end{aligned}
  \end{equation}


  \vspace{5mm}
  \centerline{\LARGE 4-3}
  \vspace{5mm}
  \textmc{\large (1)\\}
  \begin{equation}
    \begin{aligned}
        &\bm{F}=-m\omega^2x(t) -2m\beta\frac{dx(t)}{dt}\nonumber\\
    \end{aligned}
  \end{equation}

  \textmc{\large (2)\\}
  \begin{equation}
    \begin{aligned}
        &\omega を全て \beta に置き換える。\nonumber\\
        &m\frac{d^2x(t)}{dt^2} = -m \beta^2 x(t) -2m\beta \frac{dx(t)}{dt}\nonumber\\
        &m>0より、\nonumber\\
        &\frac{d^2x(t)}{dt^2} = - \beta^2 x(t) -2\beta \frac{dx(t)}{dt}\hspace{3mm} \cdots ①\nonumber\\
        &x(t)=e^{\lambda t}を代入\nonumber\\
        &\lambda^2e^{\lambda t} = -\beta^2e^{\lambda t} -2\beta\lambda e^{\lambda t}\nonumber\\
        &e^{\lambda t}>0より、\nonumber\\
        &\lambda^2 + 2\beta\lambda + \beta^2 = 0\nonumber\\
        &(\lambda + \beta)^2 = 0\nonumber\\
        &\lambda + \beta > 0より、\nonumber\\
        &\lambda + \beta = 0\nonumber\\
        &\lambda = -\beta \nonumber\\
        &\therefore x(t) = e^{-\beta t}\nonumber\\
    \end{aligned}
  \end{equation}

  \textmc{\large (3)\\}
  \begin{equation}
    \begin{aligned}
        &(2)より、x(t) = Ce^{-\beta t} \hspace{3mm} (Cは定数関数)\nonumber\\
        &Cをy(t)という関数に置き換えてみる。\nonumber\\
        &x(t) = y(t)e^{-\beta t}と仮定して、(2)の①の式に代入して解いていく。\nonumber\\
        &\frac{d^2 \hspace{1mm} y(t)e^{-\beta t}}{dt^2} = -\beta^2y(t)e^{-\beta t} -2\beta\frac{d \hspace{1mm} y(t)e^{-\beta t}}{dt}\nonumber\\
        &\frac{d^2y(t)}{dt^2}e^{-\beta t} + (\beta^2 -\beta^2)y(t)e^{-\beta t} = 0\nonumber\\
        &e^{-\beta t} >0より、\nonumber\\
        &\frac{d^2y(t)}{dt^2}=0\nonumber\\
        &したがって、y(t)=C_{1}t + C_{2} \hspace{3mm} (C_{1},C_{2} \in \mathbb{R})とおける。\nonumber\\
        &\therefore x(t) =(C_{1}t + C_{2})e^{-\beta t}\nonumber\\
        &=C_{1}te^{-\beta t} +C_{2}e^{-\beta t}\nonumber\\
        &x(t)のもう一つの解は、te^{-\beta t}となる。\nonumber\\
    \end{aligned}
  \end{equation}

  \textmc{\large (4)\\}
  \begin{equation}
    \begin{aligned}
        &x(t)=C_{1}te^{-\beta t} +C_{2}e^{-\beta t}\hspace{3mm}(一般解)\nonumber\\
    \end{aligned}
  \end{equation}

  \textmc{\large (5)\\}
  \begin{equation}
    \begin{aligned}
        &運動:\hspace{1mm} 臨界減衰\nonumber\\
        & \omega > \beta の場合よりも、収束するのが早い上に、\omega > \beta の場合と異なり、振動しない。\nonumber\\
    \end{aligned}
  \end{equation}

  \textmc{\large (6)\\}
  \begin{equation}
    \begin{aligned}
      &ドアが閉まる際に、ドアの動きが振動しないようにするドアダンパーへの利用。\nonumber\\
    \end{aligned}
  \end{equation}



\end{document}