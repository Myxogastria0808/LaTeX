\documentclass[dvipdfmx,uplatex]{jsarticle}
\title{情報リテラシー(演習)第3回}
\author{
    名前: 長田悠生\\
    学籍番号: 202310330\\
}
\date{2023/6/12}

\usepackage{url}

\begin{document}
  \begin{titlepage}
    \maketitle
    \begin{center}
      \textmc{\HUGE \LaTeX}
    \end{center}
    \thispagestyle{empty}
  \end{titlepage}

  \centerline{\huge トピック: テレワーク\\}
  \vspace{10mm}
  \textmc{\LARGE 賛成派の記事\\}
  \begin{thebibliography}{99}
    \bibitem{NTT東日本}中立の立場の記事(記事1): 【まとめて紹介】テレワークにおけるメリット・デメリットとは?, \url{https://business.ntt-east.co.jp/content/telework_start/introduction/merit} \\
    \bibitem{株式会社リクルート}8割賛成の記事(記事2): やってみて良かった! 約8割がテレワーク賛成派。テレワークで感じたメリット・デメリットとは?, \url{https://www.recruit.co.jp/sustainability/iction/ser/ot/0261.html} \\
    \bibitem{株式会社スタッフサービス・ホールディングス} 中立の立場の記事(記事3): 株式会社スタッフサービス・ホールディングス \url{https://digitalpr.jp/r/50318} \\
  \end{thebibliography}

  \textmc{\large 各賛成派の記事の選択した理由\\}
  \textmc{[記事1を選択した理由]\\}
  \textmc{
     NTT東日本は、資本金3,350億円、従業員数4,900名(2022.3.31時点)の大企業である。記事の信憑性が低いことで、企業の信頼を下げることはNTT東日本にとっても大きな不利益を被ることになると容易に推定できる。
    また、データのソースもはっきりしているため、信用できる記事とした。\\
  }
  \textmc{[記事2を選択した理由]\\}
  \textmc{
     リクルートは、資本金3億5千万円、従業員数19,836人(2023年4月1日現在 / アルバイト・パート含)の大企業である。記事の信憑性が低いことで、企業の信頼を下げることはリクルートにとっても大きな不利益を被ることになると容易に推定できる。
    また、データのソースもはっきりしているため、信用できる記事とした。\\
  }
  \textmc{[記事3を選択した理由]\\}
  \textmc{
     株式会社スタッフサービス・ホールディングスは、資本金3,504億円、従業員数5,079名(2023年4月時点)の大企業である。記事の信憑性が低いことで、企業の信頼を下げることは株式会社スタッフサービス・ホールディングスにとっても大きな不利益を被ることになると容易に推定できる。
    また、データのソースもはっきりしているため、信用できる記事とした。\\
  }

  \newpage
  \textmc{\large 各賛成派の記事の要約\\}
  \textmc{[記事1の要約]\\}
  \textmc{
    この記事では、テレワークを導入するメリットについて「企業側」・「従業員側」・「社会側」の3つの視点からそれぞれ言及している。まず企業側についての利点だが、柔軟な働き方ができるため、多様な人材を雇用しやすい。また、オフィスで発生していた雑費を抑えたり、
    災害時に連携を取るための基盤を構築できたり、デジタル化の促進を促すこともできる。その結果、企業のイメージが向上などに繋がってくると書かれている。次に、従業員側の利点だが、通勤時の満員電車等でのストレスを削減することができる。また、独立した空間での作業のために集中がしやすく、
    本来は通勤時間で失われた時間を副業等に充てることができる。最後に、社会側での利点だが、テレワークが浸透することで生産性が向上し、労働人口の減少への対策や雇用創出につながると、この記事は踏んでいる。また、それに伴う経済効果も期待されている。そして、オフィスの電力消費が抑えられることで、
    環境負荷の削減にも間接的に関与していると考えられている。\\
  }
  \textmc{[記事2の要約]\\}
  \textmc{
    この記事が行ったアンケートによると、実際にテレワークに切り替えてみて良かったと回答した人は(「どちらかといえば良かった」も含め)、78.8\%に達したそうだ。また、テレワークで感じたメリットについて問ったアンケートでは、
    通勤時間を有効に活用できるという声が多かったそうだ。具体的には、朝の時間にゆとりを持てるようになった。混雑した電車に乗らなくて済むなどの声が上がっていた。そのほかにも様々なメリットの声が掲載されていた。\\
  }
  \textmc{[記事3の要約]\\}
  \textmc{
    この記事で行ったインターネット調査によると、テレワーク賛成派が76.3\%に登った。20~30代の若年層で賛成傾向になったそうだ。
    主な賛成理由としては、「生産性やワークライフバランスの向上」「通勤時間の削減」が挙げられている。\\
  }

  \newpage
  \textmc{\LARGE 反対派の記事\\}
  \begin{thebibliography}{99}
    \bibitem{NTT東日本}中立の立場の記事(記事1): 【まとめて紹介】テレワークにおけるメリット・デメリットとは?, \url{https://business.ntt-east.co.jp/content/telework_start/introduction/merit} \\
    \bibitem{株式会社リクルート}2割反対の記事(記事2): やってみて良かった! 約8割がテレワーク賛成派。テレワークで感じたメリット・デメリットとは?, \url{https://www.recruit.co.jp/sustainability/iction/ser/ot/0261.html} \\
    \bibitem{株式会社スタッフサービス・ホールディングス} 中立の立場の記事(記事3): 株式会社スタッフサービス・ホールディングス \url{https://digitalpr.jp/r/50318} \\
  \end{thebibliography}

  \textmc{\large 各反対派の記事の選択した理由\\}
  \textmc{[記事1を選択した理由]\\}
  \textmc{
     NTT東日本は、資本金3,350億円、従業員数4,900名(2022.3.31時点)の大企業である。記事の信憑性が低いことで、企業の信頼を下げることはNTT東日本にとっても大きな不利益を被ることになると容易に推定できる。
    また、データのソースもはっきりしているため、信用できる記事とした。\\
  }
  \textmc{[記事2を選択した理由]\\}
  \textmc{
     リクルートは、資本金3億5千万円、従業員数19,836人(2023年4月1日現在 / アルバイト・パート含)の大企業である。記事の信憑性が低いことで、企業の信頼を下げることはリクルートにとっても大きな不利益を被ることになると容易に推定できる。
    また、データのソースもはっきりしているため、信用できる記事とした。\\
  }
  \textmc{[記事3を選択した理由]\\}
  \textmc{
     株式会社スタッフサービス・ホールディングスは、資本金3,504億円、従業員数5,079名(2023年4月時点)の大企業である。記事の信憑性が低いことで、企業の信頼を下げることは株式会社スタッフサービス・ホールディングスにとっても大きな不利益を被ることになると容易に推定できる。
    また、データのソースもはっきりしているため、信用できる記事とした。\\
  }

  \textmc{\large 各反対派の記事の要約\\}
  \textmc{[記事1の要約]\\}
  \textmc{
    この記事では、テレワークを導入するデメリットについて「企業側」・「従業員側」の2つの視点からそれぞれ言及している。まず、企業側についての欠点だが、テレワーク中の従業員の勤怠管理が難しいことが挙げられる。
    また、今までは同じオフィスに勤務していたためにプロジェクトやタスクの進捗管理が簡単に行えたが、テレワークになるとプロジェクトやタスクの進捗管理そのものに割く時間が増える。そして、セキュリティリスクについても考えなければならない。
    最後に、従業員側についての欠点だが、対面での意思疎通ができないためにコミュニケーション不足に陥りやすい傾向にある。また、時間や作業場所の確保等を全て自分で行わないといけないために、作業効率が低下しやすい傾向にある。
    自宅にとどまる時間が長くなるため、運動不足にも気を付けなければならない。\\
  }
  \textmc{[記事2の要約]\\}
  \textmc{
    この記事では、テレワークで感じたデメリットのかアンケート結果を掲載している。この記事の掲載している結果によると、「運動不足になる」・「社内コミュニケーションが減った」・「プリンターなどがなく紙の書類のやり取りができない」の3つが上位に挙げられていた。
    在宅勤務のために、運動する機会がなくなっていしまい、運動不足に感じている人が増えているみたいだ。また、チャットアプリでの会話だけでは、コミュニケーション不足を感じてしまう人もいたようだ。また、自宅に必ず機材がそろっているわけではないのも、テレワークの欠点と
    なってしまっていると、この記事では説明している。

  }
  \textmc{[記事3の要約]\\}
  \textmc{
    この記事で行ったインターネット調査によると、テレワークが導入されていない人の約9割が反対という結果になったそうだ。反対の主な理由としては、
    業種業態的にそもそも導入ができない等の意見が掲載されていた。\\
  }

  \begin{thebibliography}{99}
    \bibitem{NTT東日本}NTT東日本の概要についての引用: プロフィール, \url{https://www.ntt-east.co.jp/aboutus/profile.html} \\
    \bibitem{株式会社リクルート}株式会社リクルートの概要についての引用: 会社概要, \url{https://www.recruit.co.jp/company/profile/} \\
    \bibitem{株式会社スタッフサービス・ホールディングス}株式会社スタッフサービス・ホールディングスの概要についての引用: 会社情報, \url{https://www.staffservice.co.jp/company/info/data.html} \\
  \end{thebibliography}

  \newpage
  \centerline{\huge 「剽窃論文,盗用論文」の具体的な事例\\}
  \vspace{10mm}
  \textmc{\LARGE 韓国人の少年が起こした盗用についての記事\\}
  \begin{thebibliography}{99}
    \bibitem{ヤフー株式会社}YAHOO!ニュース 「天才少年」が起こした論文盗作事件(記事4), \url{https://news.yahoo.co.jp/byline/enokieisuke/20151128-00051908} \\
  \end{thebibliography}
  \textmc{[記事4が信頼できる理由]\\}
  \textmc{
     ヤフー株式会社は、資本金300百万円(2022念3付き31日)、従業員数7,597名(2022年3月31日)の大企業である。記事の信憑性が低いことで、企業の信頼を下げることはヤフー株式会社にとっても大きな不利益を被ることになると容易に推定できる。
    また、データのソースもはっきりしている。また、この記事の著者である榎木英介は、ジャーナリズムとしていくつかの有名な省を受賞している。
    これらの理由から、この記事は信頼できると考えた。\\
  }
  \textmc{[記事4に書かれている不正の具体的な内容]\\}
  \textmc{
    2015年に発表されたソン氏とパク氏の論文の内容に、パク氏自身が以前に発表した論文の引用が見られたが、引用したことを明示しなかったことで、
    Astrophysical Journalに掲載されていた当該論文が撤回されることになった。\\
  }

  \vspace{10mm}
  \textmc{\LARGE 同志社大学元大学院生の盗用についての記事\\}
  \begin{thebibliography}{99}
    \bibitem{文部科学省}同志社大学元大学院生による研究活動上の不正行為(盗用)の認定について(記事5), \url{https://www.mext.go.jp/a_menu/jinzai/fusei/1421623.htm} \\
  \end{thebibliography}
  \textmc{[記事5が信頼できる理由]\\}
  \textmc{
    文部科学省は、公的な国の機関であるため、事実確認の取れない情報を掲載するすることはないだろうと推定した。万が一、虚偽の情報を掲載していた場合、
    国の信頼を失墜させるとても野蛮な行いといえよう。
  }
  \textmc{[記事5に書かれている不正の具体的な内容]\\}
  \textmc{
    下記の内容は、\url{https://www.mext.go.jp/a_menu/jinzai/fusei/1421623.htm}のサイトの【不正事案の概要等】の1.告発内容及び調査結果の概要を引用したものである。\\
    \\
    -----引用-----\\
     同志社大学大学院社会学研究科に在学していた大学院生が平成25年に書籍に寄稿した論文について、平成30年9月13日、同年9月25日、平成31年1月23日に、各々別の告発者から研究活動上の不正行為(盗用等)を行ったとする告発がなされた。予備調査の結果を受けて本調査を行うこととし、研究倫理委員会の下に専門調査委員会を設置した。告発された内容について調査・検証した結果、盗用(特定不正行為)があったと認定した。\\
    -------------\\

  }
  \begin{thebibliography}{99}
    \bibitem{ヤフー株式会社}ヤフー株式会社の概要についての引用: 会社概要, \url{https://about.yahoo.co.jp/info/company/} \\
  \end{thebibliography}
\end{document}