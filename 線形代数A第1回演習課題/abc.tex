\documentclass[dvipdfmx,uplatex]{jsarticle}
\title{線形代数A第1回演習課題}
\author{
    名前: 長田悠生\\
    学籍番号: 202310330
    }
\date{2023/5/26}

\usepackage{amsmath,amssymb}
\usepackage{amsthm}
\usepackage{bm}

\begin{document}
  \begin{titlepage}
    \maketitle
    \begin{center}
      \textmc{\HUGE \LaTeX}
    \end{center}
    \thispagestyle{empty}
  \end{titlepage}

  \textmc{自宅にスキャナーがなく、また、カメラの性能が悪いために何が書いているか読み取れそうにないので、\LaTeX で提出します。あらかじめご了承ください。 \\}
  \centerline{\large 問1}
  \vspace{10mm}
  \textmc{写像$f:\mathbb{N}\longrightarrow\mathbb{N}$を\\}
  \begin{equation}
    f(n) = 
    \left \{
        \begin{aligned}
            &3n + 1 (nが奇数のとき),\\
            &\frac{n}{2} (nが偶数のとき)
        \end{aligned}
    \right. \nonumber
  \end{equation}
  \textmc{と定義する.このとき, \\}
  \textmc{\large (a)$f$が全射であることを示せ. \\}
  \begin{proof}
    \begin{flushleft}
        $f$が全射であることの証明\\
        任意の実数$y$に対して、
        \begin{equation}
          y = 
          \left \{
              \begin{aligned}
                  &3n + 1 (nが奇数のとき)\\
                  &\frac{n}{2} (nが偶数のとき)
              \end{aligned}
          \right. \nonumber
        \end{equation}
        とおく。\\
        $y=3n+1$($n$が奇数のとき)について、$n$が奇数より、\\
        $m\in\mathbb{N}$かつ$n=2m+1$である$m$を定義する。\\
        \begin{equation}
          \begin{aligned}
            y &= 3(2m+1)+1\nonumber\\
            &= 6m+4\nonumber\\
            &= 2(3m+2)\nonumber
          \end{aligned}
        \end{equation}
        よって、$y$は偶数である。\\
        また、$y=\frac{n}{2}$($n$が偶数のとき)について、$n$が偶数より、\\
        $m\in\mathbb{N}$かつ$n=2m$である$m$を定義する。\\
        \begin{equation}
          \begin{aligned}
            y &= \frac{2m}{2}\nonumber\\
            &= m\nonumber
          \end{aligned}
        \end{equation}
        よって、$y$は自然数である。\\
        $\therefore 写像f:\mathbb{N}\longrightarrow\mathbb{N}$より、終域の全てを$f(n)$の地域は表している。それは、$f$が全射であることを意味する。
    \end{flushleft}
\end{proof}

\newpage
\textmc{\large (b)$f$が単射であることを示せ. \\}
\begin{proof}
  \begin{flushleft}
    反例を以下に示す。\\
    写像$f({n}_{1})、f({n}_{2})$を、${n}_{1}$を奇数、${n}_{2}$を偶数とおく。\\
    \begin{equation}
      \begin{aligned}
        &f({n}_{1}=1) = 3 \times 1+1 = 4\nonumber\\
        &f({n}_{2}=8) = \frac{8}{2} = 4\nonumber\\
        &f({n}_{1}=1) = f({n}_{2}=8) より、fは単射ではない。
      \end{aligned}
    \end{equation}
  \end{flushleft}
\end{proof}

  \centerline{\large 問2}
  \textmc{
    $\bm{a}=\begin{pmatrix} 2 \\ 4 \\ 3 \end{pmatrix}, \bm{b}=\begin{pmatrix} 5 \\ 9 \\ 7 \end{pmatrix}$
    の両方に直交する、長さ1のベクトルを$\bm{c}$とすると、\\
  }
  \textmc{$(\bm{a},\bm{c})=0かつ(\bm{b},\bm{c})=0で\|\bm{c}\|=1、\bm{c}=\begin{pmatrix} x \\ y \\ z \end{pmatrix}$とおくと、\\}
  \begin{equation}
    \begin{aligned}
      \\
      &(\bm{a},\bm{c})=2x + 4y + 3z = 0 \cdots ① \nonumber\\
      &(\bm{b},\bm{c})=5x + 9y + 7z = 0 \cdots ② \nonumber\\
      &\|\bm{c}\|^2=x^2 + y^2 + z^2 = 1 \cdots ③ \nonumber\\
      \\
      &z = -2y \cdots ①\times 5 - ② \times 2\\
      &x = y \cdots ①\times 7 - ② \times 3\\
      &z = -2y = -2x \cdots ④\\
      &③に④を代入\\
      &x^2 + x^2 + (-2x)^2 = 1\\
      &x = \pm \frac{\sqrt{6}}{6}\\
      \\
      &\therefore \bm{c}=\begin{pmatrix} \frac{\sqrt{6}}{6} \\ \\ \frac{\sqrt{6}}{6} \\ \\ -\frac{\sqrt{6}}{3} \end{pmatrix} \quad or \quad \bm{c}=\begin{pmatrix} -\frac{\sqrt{6}}{6} \\ \\ -\frac{\sqrt{6}}{6} \\ \\ \frac{\sqrt{6}}{3} \end{pmatrix}
    \end{aligned}
  \end{equation}

  \newpage
  \centerline{\large 問3}
  \begin{proof}
    \begin{flushleft}
      シュワルツの不等式の証明\\
      \begin{equation}
        \begin{aligned}
          &\bm{a}=\begin{pmatrix} {a}_{1} \\ {a}_{2} \end{pmatrix}、\bm{b}=\begin{pmatrix} {b}_{1} \\ {b}_{2} \end{pmatrix}とする。\nonumber\\
          &\|\bm{a}\| \|\bm{b}\| \geq 0、\|(\bm{a},\bm{b})\| \geq 0より共に2乗すると、\nonumber\\
          &( \|\bm{a}\| \|\bm{b}\| )^2 - \|(\bm{a},\bm{b})\|^2\nonumber\\
          &=({a}_{1}^2 + {a}_{2}^2)({b}_{1}^2 + {b}_{2}^2) - ({a}_{1} {b}_{1} + {a}_{2} {b}_{2})^2\nonumber\\
          &= {a}_{1}^2 {b}_{2}^2 + {a}_{2}^2 {b}_{1}^2 - 2 {a}_{1} {b}_{1} {a}_{2} {b}_{2}\nonumber\\
          &= ({a}_{1} {b}_{2} - {a}_{2} {b}_{1})^2 \geq 0\nonumber\\
          &({a}_{1} {b}_{2} = {a}_{2} {b}_{1}のとき、等号成立)\nonumber
        \end{aligned}
      \end{equation}
    \end{flushleft}
  \end{proof}

  \centerline{\large 問4}
  \begin{equation}
    \begin{aligned}
      &\bm{a}=\begin{pmatrix} {a}_{1} \\ {a}_{2} \end{pmatrix}、\bm{x}=\begin{pmatrix} {x}_{1} \\ {x}_{2} \end{pmatrix}とする。\nonumber\\
      \\
      & \left( \bm{a},\bm{x}-(\bm{x},\bm{a})\frac{\bm{a}}{\|\bm{a}\|^2} \right) \nonumber\\
      &= {a}_{1} \left( {x}_{1} - {a}_{1} \times \frac{({a}_{1} {x}_{1} + {a}_{2} {x}_{2})}{{a}_{1}^{\hspace{1mm}2} + {a}_{2}^{\hspace{1mm}2}} \right) + {a}_{2} \left( {x}_{2} - {a}_{2} \times \frac{({a}_{1} {x}_{1} + {a}_{2} {x}_{2})}{{a}_{1}^{\hspace{1mm}2} + {a}_{2}^{\hspace{1mm}2}} \right)\nonumber\\
      &= \frac{{x}_{1} {a}_{1} {a}_{2}^{\hspace{1mm}2} - {x}_{1} {a}_{1}^{\hspace{1mm}2} {a}_{2} + {x}_{1} {a}_{1}^{\hspace{1mm}2} {a}_{2} - {x}_{1} {a}_{1} {a}_{2}^{\hspace{1mm}2}}{{a}_{1}^2 + {a}_{2}^2} = 0 \nonumber\\
      \\
      &\therefore \bm{a}と\bm{x}-(\bm{x},\bm{a})\frac{\bm{a}}{\|\bm{a}\|^2}は垂直である。\nonumber
    \end{aligned}
  \end{equation}
  
\end{document}