\documentclass[dvipdfmx,uplatex]{jsarticle}
\title{確率論第6回課題}
\author{
    学籍番号: 202310330\\
    名前: 長田悠生
    }
\date{2023/6/4}

\usepackage{amsmath,amssymb}

\begin{document}
  \begin{titlepage}
    \maketitle
    \begin{center}
      \textmc{\HUGE \LaTeX}
    \end{center}
    \thispagestyle{empty}
  \end{titlepage}

  \centerline{\huge 演習課題6A}
  \vspace{10mm}
  \textmc{確率変数をX、確率質量関数を${p}_{X}(x)$としたとき、期待値は以下の式で表せる。}
  \begin{equation}
    \sum_{x} x{p}_{X}(x)\nonumber\\
  \end{equation}
  \textmc{
    ${p}_{X}(xi)$を$k$回目までに目標の当たりの数-1の当たりが出てから$k+1$回目に目標の当たりの数が出る確率だとすると、
    ${p}_{X}(xi)$は幾何分布となり、期待値は$試行回数 \times {p}_{X}(xi + 1)$より、$xi$までの試行回数を$k$と置いているので、
  }
  \begin{equation}
    \begin{aligned}
        &E[Xi] = \sum_{k=0}^\infty (k+1)p(1-p)^k\nonumber\\
        &=p \times \sum_{k=0}^\infty (k+1)(1-p)^k\nonumber\\
        &=p \left\{ \lim_{n \to \infty} \sum_{k=0}^n (k+1)(1-p)^k \right\}\nonumber\\
        &=p \left\{ \lim_{n \to \infty} \frac{1-(1-p)^{n+1}}{p^2} -\lim_{n \to \infty} \frac{(n+1)(1-p)^{n+1}}{p} \right\}\nonumber\\
        &= \frac{1}{p}\nonumber\\
        \\
        &E[X]=E[X1] + E[X2] + \cdots + E[Xr]\nonumber\\
        &=r \times \frac{1}{p} = \frac{r}{p}\nonumber\\
        &\therefore E[X]=\frac{r}{p}\nonumber\\
    \end{aligned}
  \end{equation}

  \newpage
  \centerline{\huge 演習課題6B}
  \vspace{10mm}
  \textmc{\large (a)\\}
  \begin{equation}
    \begin{aligned}
        &\int_{\infty}^{-\infty} f(x) dx = 1 より、 \nonumber\\
        &\int_{\infty}^{-\infty} {f}_{X}(x) dx = \int_{-\infty}^{0} {f}_{X}(x) dx + \int_{0}^{\infty} {f}_{X}(x) dx = 1 \nonumber\\
        &=0 + \int_{0}^{\infty} \lambda e^{-\frac{x}/100} dx \nonumber\\
        &=0 + \left[ -100 \lambda e^{-\frac{x}/100}  \right]_0^\infty\nonumber\\
        &100\lambda=1\nonumber\\
        &\therefore \lambda=\frac{1}{100}\nonumber\\
    \end{aligned}
\end{equation}

\textmc{\large (b)\\}
\begin{equation}
  \begin{aligned}
    &0 \leq x \leq 100の範囲の{f}_{X}(x)についての積分を行えばよいので、\nonumber\\
    &\int_{0}^{100} f(x) dx = \int_{0}^{100} \lambda e^{-\frac{x}/100} dx\nonumber\\
    &= \left[ -100 \lambda e^{-\frac{x}/100}  \right]_0^100 \nonumber\\
    &= 100\lambda - \frac{100\lambda}{e}\nonumber\\
    &\lambda = \frac{1}{100}より、\nonumber\\
    &=1-\frac{1}{e}\nonumber\\
    &\therefore 1-\frac{1}{e}\nonumber\\
  \end{aligned}
\end{equation}
\end{document}