\documentclass[dvipdfmx,uplatex]{jsarticle}
\usepackage{tree}
\title{確率論第2回演習課題}
\author{長田悠生}
\date{2023/4/25}

\begin{document}
  \begin{titlepage}
    \maketitle
    \begin{center}
      \textmc{\HUGE \LaTeX}
    \end{center}
    \thispagestyle{empty}
  \end{titlepage}

  \centerline{\huge 演習課題2A}
  \vspace{10mm}
  \textmc{問題の理解のために、まずツリー構造に直してみる。 \\}
  \centerline{
    \ROOT{被験者がコインを投げる}{
      \BRANCH{コインの裏}{\LEAF{pattern3 被験者が「はい」と答える}}
      \BRANCH{コインの表}{\LEAF{pattern1 被験者が「はい」と答える}\LEAF{pattern2 被験者が「いいえ」と答える}}
  }}
  \textmc{
    問題文をツリー構造に直すと、結果にはpattern1から3があることがわかった。
    次に、pattern1から3について、集合A・Bを用いて表す。 \\
  }
  \textmc{\large Pattern1}
  \begin{equation}
    P(A \cap B)
  \end{equation}
  \textmc{\large Pattern2}
  \begin{equation}
    P(A^c \cap B)
  \end{equation}
  \textmc{\large Pattern3}
  \begin{equation}
    P(B^c)
  \end{equation}
  \textmc{
    AとBは互いに独立の関係で、AとBのみでCを表現できる。
    よって、上図のツリー構造より$P(C)$は、 \\
  }
  \begin{equation}
    P(A \cap B) + P(B^c) = P(C)
  \end{equation}
  \textmc{と表せる。\\}
  \vspace{3mm}
  \textmc{よって、$P(A)$は、\\}
  \begin{equation}
    \frac{1}{2}P(A) + \frac{1}{2} = \frac{55}{100}
  \end{equation}
  \begin{equation}
    P(A) = 0.1
  \end{equation}

  \newpage
  \centerline{\huge 演習課題2B}
  \vspace{10mm}
  \textmc{事象AとBが独立より、独立の定義から次のことが成り立つ。\\}
  \begin{equation}
    P(A \cap B) = P(A)P(B)
  \end{equation}
  \textmc{事象Bと事象$B^c$の和集合は、全体集合を表すので、\\}
  \begin{equation}
    P(B) + P(B^c) = 1
  \end{equation}
  \textmc{また、$P(A \cap B)$と$P(A \cap B^c)$は排反より、\\}
  \begin{equation}
    P(A \cap B) + P(A \cap B^c) = P(A)
  \end{equation}
  \textmc{上記の3式を利用して、事象Aと事象$B^c$が独立であることを示す。\\}
  \textmc{以下、事象Aと事象$B^c$が独立であることの証明\\}
  \begin{equation}
    P(A \cap B) = P(A) - P(A \cap B^c)
  \end{equation}
  \begin{equation}
    P(B) = 1 - P(B^c)
  \end{equation}
  \begin{equation}
    P(A) - P(A \cap B^c) = \{1 - P(B^c)\}P(A)
  \end{equation}
  \begin{equation}
    P(A) - P(A \cap B^c) = P(A) - P(B^c)P(A)
  \end{equation}
  \begin{equation}
    P(A \cap B^c) = P(B^c)P(A)
  \end{equation}
  \textmc{式(14)は、事象Aと$B^c$が独立であるときの、定義を表す式である。\\}
  \textmc{よって、証明を得る。\\}

\end{document}