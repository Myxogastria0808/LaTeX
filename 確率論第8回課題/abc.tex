\documentclass[dvipdfmx,uplatex]{jsarticle}
\title{
}
\author{
    名前: 長田悠生\\
    学籍番号: 202310330\\
}
\date{2023/6/18}

\usepackage{amsmath,amssymb}
\usepackage{bm}
\usepackage{mathtools}

\begin{document}
  \begin{titlepage}
    \maketitle
    \begin{center}
      \textmc{\HUGE \LaTeX}
    \end{center}
    \thispagestyle{empty}
  \end{titlepage}

  \centerline{\huge 演習課題8A}
  \vspace{10mm}
  \begin{equation}
    \begin{aligned}
        &面積が \pi であるので、面積が1になるように調整する。\nonumber\\
        &すると、{f}_{X,Y}(x,y) = \frac{1}{\pi}\nonumber\\
        &f_{Y}(y) = \int_{-\infty}^{\infty} f_{X,Y}(x,y) dx = \int_{-\sqrt{1-y^2}}^{\sqrt{1-y^2}} \frac{1}{\pi} dx = \frac{2}{\pi}\sqrt{1-y^2}\nonumber\\
        &f_{X|Y}(x|y) = \frac{\frac{1}{\pi}}{\frac{2\sqrt{1-y^2}}{\pi}} = \frac{1}{2\sqrt{1-y^2}}\nonumber\\
        &\therefore f_{Y}(y) = \frac{2}{\pi}\sqrt{1-y^2}、f_{X|Y}(x|y) = \frac{1}{2\sqrt{1-y^2}}\nonumber\\
    \end{aligned}
  \end{equation}

  \vspace{10mm}
  \centerline{\huge 演習課題8B}
  \vspace{10mm}
  \begin{equation}
    \begin{aligned}
        &一様分布に従うより、f_{X}(x) = \frac{1}{n}\nonumber\\
        &E \left[ g(x) \right] = \int_{-\infty}^{\infty} g(x)f_{X}(x) dx より、\nonumber\\
        &g(x) = |X - m|とおくと、以下のように表せる。\nonumber\\
        &E \left[ |X - m| \right] = \int_{0}^{n}|X - m|f_{X}(x)dx\nonumber\\
        &=\frac{m^2}{n} -m + \frac{n}{2}\nonumber\\
        &f(m)=\frac{m^2}{n} -m + \frac{n}{2}の最小値の際のmの値を求めればいいので、\nonumber\\
        &=\frac{1}{n} \left( m - \frac{n}{2} \right)^2 + \frac{n}{4}\nonumber\\
        &m=\frac{n}{2}のとき、最小値が\frac{n}{4}となる。\nonumber\\
        &\therefore m=\frac{n}{2}のとき、E \left[ |X - m| \right]が最小。\nonumber\\
    \end{aligned}
  \end{equation}

\end{document}